\documentclass[runningheads,a4paper]{llncs}

\usepackage{amssymb}
\setcounter{tocdepth}{3}
\usepackage{graphicx}

\usepackage{url}
\urldef{\mailsa}\path|galk@cs.biu.ac.il|
\newcommand{\keywords}[1]{\par\addvspace\baselineskip
\noindent\keywordname\enspace\ignorespaces#1}

\begin{document}

\mainmatter  % start of an individual contribution

\title{Robocup 2009\\Standard Platform League\\Team BURST Description}

\titlerunning{SPL Team BURST Description}

\author{Eran Polosetski$^1$\and Vladimir (Vova) Sadov$^1$\and
Alon Levy$^1$\and Asaf Shilony$^1$\and Anat Sevet$^1$\and Elad Alon$^1$\and Meytal Traub$^1$ \and Gal Kaminka$^1$ \and Eli Kolberg$^2$ }
%
\authorrunning{SPL Team BURST Description}
% (feature abused for this document to repeat the title also on left hand pages)

% the affiliations are given next; don't give your e-mail address
% unless you accept that it will be published
\institute{$^1$Computer Science \& $^2$Computer Engineering Departments,\\
              Bar Ilan University, Israel\\
\mailsa\\
\url{http://shwarma.cs.biu.ac.il/robocup/}}

%
% NB: a more complex sample for affiliations and the mapping to the
% corresponding authors can be found in the file "llncs.dem"
% (search for the string "\mainmatter" where a contribution starts).
% "llncs.dem" accompanies the document class "llncs.cls".
%

\toctitle{Robocup 2009 SPL}
\tocauthor{Team BURST Description}
\maketitle


\begin{abstract}
Team BURST is a newly-formed team which intends to compete in the RoboCup
2009 standard platform league. BURST is the first non-Junior RoboCup team from Israel, ever. Our main research interests within the scope of the SPL are \emph{Robust, realtime vision}, \emph{Machine learning for gait generation and adaptation} and \emph{Architectures for humanoid decision-making}.


\keywords{Humanoids, Nao, Robotic soccer.}
\end{abstract}


\section{Introduction}
Team BURST (Bar-Ilan University Robotic Soccer Team) is a newly-formed team which is competing in the RoboCup 2009 standard platform league. BURST is the first senior (non-Junior) RoboCup team from Israel, ever. The team is composed of two previously-independent RoboCup efforts at Bar Ilan University, lead by Kaminka (Computer Science department) and Kolberg (Computer Engineering department). Both team leaders have considerable experience in RoboCup (as participants, organizers, league chairs, and symposium co-chairs). All student members of the team have relevant experience in robotics. 

We have a long history of publications, in and out of RoboCup forums, and in
and out of robotic soccer (dozens of publications in the last three years, alone). We
take RoboCup-based research very seriously, and have focused research goals that we
hope to achieve by utilizing the robotic platform, and our participation in the league.
Specifically, we are interested in: (i) anytime object-recognition (as a basis for anytime
visual SLAM); (ii) using reinforcement learning to generate robust and speedy biped
and quadped gaits; and (iii) decision-making architectures for humanoid robots.

We hope that by participating in the standard platform league, we would be able to attract Israeli fans and researchers to take a more active part in RoboCup in the coming years.



\section{Software Architecture}
Our software architecture is decomposed into components of vision (see \ref{vision}), localization (see \ref{localization}), actuation, communication and high level behavior (see \ref{behavior}). They are implemented as follows:
\begin{itemize}
\item \emph{Actuation, Low level Sensors, Walking} - We use the NaoQi client-server architecture (provided by Aldebaran) to access the lowest level modules and the actuation: namely, joint movement and walking. The actual motions we use were generated by the Choreographe tool and by reverse engineering the open loop walk engine of Aldebaran.
\item \emph{Communication} - A C++ module running in the NaoQi process exports all relevant variables to shared memory (we needed the speed for out of process on the same robot communication), while another module records these results to a gzipped file. We use ALMemory/Shared Memory/sockets to communicate, using the best means for each option.
\end{itemize}

\subsection{Vision}
\label{vision}
We use vision code courtesy of Northern Bites \cite{northern,northern-repo} with few changes, including improved color calibration tools and ball recognition, which will be contributed back. Basically this means the usual colorspace $\rightarrow$ colorset, run length encoding, blobing and object detection. Their implementation is a single module including both vision and localization code, which made it necessary for us to export the required variables through ALMemory / Shared memory.

\subsection{Localization}
\label{localization}
We use localization code courtesy of Northern Bites \cite{northern,northern-repo} with no planned changes.

\subsection{Locomotion}
Our ultimate goal is to create a robust closed-loop locomotion controller which will be able to adjust itself to the changing terrain conditions, to switch gaits dynamically and prevent falls. To achieve this goal we have started from the basic NaoQi open-loop controller and generated several stable gaits variations with NaoQi built-in parameters. Using the resulting scripts we extracted command sequences for each joint involved in walking and built our own open-loop locomotion controller fully adjustable for our needs. We hope that FSR data, accelerometers/gyro sensors data and joints sensors position errors data will allow us to built a closed-loop controller on top of our current, open-loop locomotion controller.

\subsection{Motion Path Planning}
A considerable effort has been made on studying motion path planning. Given starting coordinates (X,Y,Yaw), target coordinates (X,Y,Yaw) and set of all the gaits defined for the robot, we are building motion planner that will propose a combination of walking commands that minimizes total travel time. This project is performed by group of outstanding undergraduate students whom we are very proud to host in our lab.

\subsection{Behaviors}
\label{behavior}
Our high level behavior engine is a python script that can be run on a seperate host or on the robot itself. This fascilitates faster development as compared to a naoqi module, and by using shared memory we achieve comparable speeds. 

Behaviors are written in python, a high level language that facilitates fast development, and using a framework written especially for the robocup event, which includes an event based programming model and a deferred mechanisem (the later is based on our team's experience with \cite{twisted}).

\label{references}


\begin{thebibliography}{4}

\bibitem{url} Team BURST Homepage, \url{http://shwarma.cs.biu.ac.il/robocup/}
\bibitem{northern} Northern Bites Homepage,
\url{http://robocup.bowdoin.edu/blog/}
\bibitem{northern-repo} Northern Bites Source code Repository
\url{http://github.com/northern-bites/}
\bibitem{twisted} Twisted Matrix
\url{http://twistedmatrix.com/trac/}
\end{thebibliography}



\end{document}
